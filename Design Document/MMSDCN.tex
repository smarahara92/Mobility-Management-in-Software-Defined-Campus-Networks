\documentclass[10pt,a4paper,titlepage]{report}
\usepackage[latin1]{inputenc}
\usepackage{amsmath}
\usepackage{amsfonts}
\usepackage{amssymb}
\usepackage{makeidx}
\usepackage{graphicx}
\usepackage[]{algorithm2e}
\usepackage[]{algorithmic}
\usepackage{scrextend}
\usepackage{listings}
\author{Smarahara Vukkisila\\ 14MCMB20}
\title{Mobility Management in Software Defined Campus Networks}

\begin{document}
	\addtokomafont{labelinglabel}{\sffamily}
	\lstdefinestyle{mystyle}{
		breakatwhitespace=false,
		keepspaces=true	
	}
	\lstset{style=mystyle}
	
	\renewcommand{\caption}{\textbf{def }}
	\renewcommand{\algorithmicforall}{\textbf{foreach }}
	\renewcommand{\in}{\textbf{ in }}
	\newcommand{\eq}{\textbf{\textit{ eq }}}
	\newcommand{\gt}{ > }
	\newcommand{\lt}{ < }
	\newcommand{\CONTINUE}{\textbf{\\ continue \\ }}
	
	\maketitle
	
	\section{HostDBEvents}
		\begin{lstlisting}[language=Python]
interface HostDBEvents {

    # Host entntry
    ##############################################################
    # ---------------------------------------------------------- #
    # | IPv6 | MAC | DPID | Portnum | Timeout | last refreshed | #
    # ---------------------------------------------------------- #
    ##############################################################

    # "hostentry" is a list containing values specified as above

	void hostAdded(hostentry);

	void hostUpdated(updatehostdentry, oldhostentry)

	void hostremoved(hostentry)

}
		\end{lstlisting}
		
	\section{HostDB}
		\begin{lstlisting}[language=Python]
class HostDB {

    # hostentry
    ##############################################################
    # ---------------------------------------------------------- #
    # | IPv6 | MAC | DPID | Portnum | Timeout | last refreshed | #
    # ---------------------------------------------------------- #
    ##############################################################

	# "hostentry" means list containing values as specified above

	List hostentries = []	  # list of hostentry

	HostDBEvents eventhandler

	#############################################################
	# It perform searching based on searchby (index in the hos- #
	# tentries) and return all entries matches with the value   #
	#############################################################
	def getHostEntries(int searchby, unsigned long value);

	##########################################################
	# It perform searching based on searchby. When it encou  #
	# ters first match it return the value based on returnby #
	##########################################################
	def getHostEntryField(int searchby, 
			      unsigned long value, 
			      int returnby);


	def addHost(unsigned long ipv6, 
		    unsigned long mac, 
		    unsigned long dpid, 
		    unsigned long phyportnum, 
		    unsgined long timeout);


	def updateHost(int searchby, 
		       unsigned long searchvalue,
		       int [] updateby,
		       unsigned long [] updatedvalues);

	def removeHost(int searchby, unsigned long searchvalue);

	def refreshHostEntry(int searchby, unsigned long searchvalue);

}
		\end{lstlisting}

			\begin{algorithm}
				\caption{getHostEntries(int searchby, unsigned long value)}
				\begin{algorithmic}[1]
					\STATE $entries \leftarrow [\ ]$
					\FORALL {$entry \in hostentries$}
						\IF {$entry[searchby] \eq value$}
							\STATE $entries.append(entry)$
						\ENDIF
					\ENDFOR
					\RETURN $entries$
				\end{algorithmic}
			\end{algorithm}
			
			\begin{algorithm}
				\caption{getHostEntryField(int searchby,} \\
				\qquad \qquad \qquad \qquad \qquad \ unsigned long value, \\
				\qquad \qquad \qquad \qquad \qquad \ int returnby)
				\begin{algorithmic}[1]
					\FORALL{$entry \in hostentries$}
						\IF{$entry[searchby] \eq value$}
							\RETURN $entry[returnby]$
						\ENDIF
					\ENDFOR
					\RETURN $0$
					\STATE $\newline$
				\end{algorithmic}
			\end{algorithm}
			
			\begin{algorithm}
				\caption{addHost(unsigned long ipv6,} \\
				\qquad \qquad \qquad unsigned long mac, \\
				\qquad \qquad \qquad unsigned long dpid, \\
				\qquad \qquad \qquad unsigned long phyportnum, \\
				\qquad \qquad \qquad unsgined long timeout, \\
				\begin{algorithmic}[1]
					\STATE $hosentry \leftarrow [ipv6, mac, dpid, phyportnum, timeout, current\_time]$
					\STATE $hostentries.append(hostentry)$
					\STATE $eventhandler.hostAdded(hostentry)$
				\end{algorithmic}
			\end{algorithm}
			
			\begin{algorithm}
				\caption{updateHost(int searchby,} \\
				\qquad \qquad \qquad \ \ \ \ unsigned long searchvalue, \\
				\qquad \qquad \qquad \ \ \ \ int [] updateby, \\
				\qquad \qquad \qquad \ \ \ \ unsigned long [] updatedvalues \\
				\begin{algorithmic}[1]
					\FORALL{$hostentry \in hostentries$}
						\IF{$hostentry[searchby] \eq searchvalue$}
							\STATE $index \leftarrow 0$
							\STATE $oldentry \leftarrow hostentry.clone()$
							\FORALL{$updatevalue \in updatedvalues$}
								\STATE $hostentry[updateby[index]] \leftarrow updatevalue$
							\ENDFOR
							\STATE $hostentry[-1] \leftarrow  current\_time$
							\STATE $eventhandler.hostUpdated(oldentry, hostentry)$
						\ENDIF
					\ENDFOR
				\end{algorithmic}
			\end{algorithm}
			
			\begin{algorithm}
				\caption {removeHost(int searchby, unsigned long searchvalue)}
				\begin{algorithmic}[1]
					\STATE $hentries \leftarrow [\ ]$
					\FORALL{$hostentry \in hostentries$}
						\IF{$hostentry[searchby] \neq searchvalue$}
							\STATE $hentries.append(hostentry)$
							\CONTINUE
						\ENDIF
						\STATE $eventhandler.hostRemoved(hostentry)$
					\ENDFOR
					\STATE $hostentries \leftarrow hentries$
				\end{algorithmic}
			\end{algorithm}
			
	\section{Switch}
	\section{Address}
		\begin{lstlisting}[language=Python]
class Address {

	List multicastaddresses; # list contains IPv6 multicast, 
				 # MAC addresses pair
		
	List addresses;		 # list contains IPv6, MAC adresses pairs
		
	List defaultIPs;	# list contains different scopes 
				# of ipv6 addresses
	
	List defaultGateway;
		
	def addNewAddress(unsigned long ipv6, unsigned long mac);
		
	def removeAddress(unsigned long ipv6);
		
	########################################################
	# It matches the given value against the addresses and #
	# multicastaddresses object based on value given in va-#
	# luetype. If it matched any address it return true    #
	# else return false                                    #
	########################################################
	def matches(unsigned long value, int valuetype);
	
	########################################################
	# It returns the mac address assosiated with the given #
	# ipv6 address if it found it returns mac address else #
	# it returns 0                                         #
	########################################################
	def getMAC(unsigned long ipv6);
	
	#########################################################
	# It returns the ipv6 address assosiated with the given #
	# mac address if it found it returns ipv6 address else  #
	# it returns 0                                          #
	#########################################################
	def getIP(unsigned long mac);
}
		\end{lstlisting}
		\newpage
		
		\section{ICMPv6Handler}
			\begin{lstlisting}[language=Python]
class ICMPv6Handler {

	Address myaddress;	# Addresses are represented by this object
				# are consider to be its addresses
	
	HostDb hostdb;
	
	Network network;
	
	def process(PacketInEvent event);
	
	###################################################
	# It sends echo reply for echo request destined   #
	# addresses in myaddress object and also it sends #
	# reply messages for DHAAD messages. The reply    #
	# message contains list of home agents addresses  #
	# which are public ipv6 addresses in "defaultIPs" #
	# field of myaddress object                       #
	###################################################
	def _handle_ICMPv6_Echo_Request(Ethernet ethpkt, unsigned long dpid, unsigned int inport);
	
	
	def _handle_ICMPv6_Neighbour_Request(Ethernet ethpkt, unsigned long dpid, unsigned int inport);
	
	def isHostReachable(unsigned long ipv6addr, unsigned long dpid, unsigned int inport);

}
		\end{lstlisting}
		
		\begin{algorithm}
			\caption{process(PacketInEvent event)}
			\begin{algorithmic}[1]
				\STATE $ethpkt \leftarrow event.packet$
				\STATE $ipv6pkt \leftarrow ethpkt.find("ipv6pkt")$
				\STATE $icmppkt \leftarrow ipv6pkt.find("icmpv6")$
				\IF{$icmppkt.type \eq 133 \AND myaddress.matches(icmppkt.target)$}
					\STATE $event.reason \leftarrow OFPP\_CONTROLLER$
					\STATE $\_handle\_ICMPv6\_Neighbour\_Solicitation(ethpkt, event.dpid, event.inport)$
				\ELSIF{$(icmppkt.type \eq 128 \OR icmppkt.type \eq 144) \AND myaddress.matches(ipv6pkt.destination)$}
					\STATE $event.reason \leftarrow OFPP\_CONTROLLER$
					\STATE $_handle_ICMPv6_Echo_Request(ethpkt, event.dpid, event.inport)$
				\ENDIF
			\end{algorithmic}
		\end{algorithm}
		
		\begin{algorithm}
			\caption{\_handle\_ICMPv6\_Echo\_Request(Ethernet ethpkt, unsigned long dpid, unsigned int inport)}
			\begin{algorithmic}[1]
				\STATE $ipv6pkt = ethpkt.find("ipv6")$
				\STATE $icmppkt = ipv6pkt.find("icmpv6")$
				\IF{$ipv6.destination \textbf{is} ALL\_HOME\_AGENTS\_MULTICAST\_ADDRESS$}
					\STATE $srcip = address.defaultPublicIP()$
					\STATE $srcmac = address.getMAC(dstip)$
				\ELSE
					\STATE $srcip = ipv6pkt.destination$
					\STATE $srcmac = ethpkt.destination$
				\ENDIF
				\STATE $pkt \leftarrow createIPv6Pkt(srcmac, dstmac, srcip, dstip)$
				\IF{$icmppkt.type == 128\ \AND\ icmppkt.code == 0$}
					\STATE $icmp = ICMPv6.createEchoReply(icmppkt.Identifier, icmppkt.SequenceNo)$
				\ELSIF {$icmppkt.type == 144\ \AND\ icmppkt.code == 0$}
					\STATE $defaultIPs = address.defaultIPs.clone().remove(srcip)$
					\STATE $icmp = ICMPv6.createHomeAgentDiscoverReply(icmppkt.Identifier, defaultIPs)$
				\ENDIF
				\STATE $pkt.data \leftarrow icmp$
				\STATE $network.sendPacket(pkt, dpid, inport)$
			\end{algorithmic}
		\end{algorithm}
		
		\begin{algorithm}
			\caption{\_handle\_ICMPv6\_Neighbour\_Solicitation(Ethernet ethpkt, unsigned long dpid, unsigned int inport)}
			\begin{algorithmic}[1]
				\STATE $ipv6pkt \leftarrow ethpkt.find("icmpv6")$
				\STATE $icmppkt \leftarrow ipv6pkt.find("icmpv6")$
				\STATE $srcip \leftarrow icmppkt.target$
				\STATE $srcmac \leftarrow myaddress.getMac(srcip)$
				\IF {$ipv6pkt.source\ \textbf{is}\ \textbf{UNSPECIFIED}$}
					\STATE $dstip = ALL\_NODES\_MULTICAST\_ADDRESS$
					\STATE $dstmac = ALL\_NODES\_MULTICAST\_ETHERNET\_ADDRESS$
				\ELSE
					\STATE $dstip = ipv6pkt.source$
					\STATE $dstmac = icmppkt.sourceLinkLayerAddress$
				\ENDIF
				\STATE $pkt \leftarrow createNeighbourAdvertisement(srcmac, dstmac, srcip, dstip, srcip, srcmac)$
				\STATE $network.sendPacket(pkt, dpid, inport)$
			\end{algorithmic}
		\end{algorithm}
	
	\section{HomeAgent}
		\begin{lstlisting}[language=Python]
class HomeAgent {
	
	List mobileNodes;
	
	ICMPv6Handler icmphandler;
	
	def _handle_Binding_Updates(PacketInEvent event);
	
}
		\end{lstlisting}
		
		\begin{algorithm}
			\caption{\_handle\_Binding\_Updates(PacketInEvent event)}
			\begin{algorithmic}[1]
				\STATE $homeaddr \leftarrow extractHomeAddress(ipv6pkt)$
				\STATE $altercaddr \leftarrow extractAlterAddress(ipv6pkt)$
				\STATE $lifetime \leftarrow extractlifetime(ipv6pkt)$
				\IF {$lifetime \eq 0$}
					\STATE $remove entry in mobilenodes list$
					\STATE $remove ip, mac address pair from myaddress$
					\STATE $reset entry in hostdb$
				\ELSE 
					\IF {$ENTRY\ EXIST$}]
						\STATE $refresh entry in hostdb$
						\STATE $update it$
						\STATE $IF ANY CHANGE UPDATE THOSE THINGS$
					\ELSE 
						\STATE $test for host reachable$
						\STATE $update entry in hostdb with default gateway fields and lifetime set to life time in mobility header$
						\STATE $add ip, mac address myaddress list$
						\STATE $add new entry to the mobilenodes list$
					\ENDIF 
				\ENDIF
			\end{algorithmic}
		\end{algorithm}
		
\end{document}
