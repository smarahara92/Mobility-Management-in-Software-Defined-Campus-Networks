\documentclass[10pt,a4paper,titlepage]{report}
\usepackage[latin1]{inputenc}
\usepackage{amsmath}
\usepackage{amsfonts}
\usepackage{amssymb}
\usepackage{makeidx}
\usepackage{graphicx}
\usepackage[]{algorithm2e}
\usepackage[]{algorithmic}
\usepackage{scrextend}
\usepackage{listings}
\author{Smarahara \\ 14MCMB20}
\title{Mobility Management in Software Defined Campus Networks}
\begin{document}
	\maketitle
	\addtokomafont{labelinglabel}{\sffamily}
	\begin{labeling}{LocalNetwork}
		\item [Address] It represent logical and physical addresses of a system. It maintains IPv6, MAC address pair
		\item [HostDB] It maintains the current position of each system in the network
		\item [Switch] It is a logical representation of a switch in the network
		\item [LocalNetwork] It has the network topology of the current network
		\item [ICMPv6] It handles the ICMPv6 messages received at the controller
		\item [HomeAgent] It handles IPv6 packets, which contains mobility header
		\item [Init] It is the start point of the software
	\end{labeling}

	\renewcommand{\caption}{\textbf{def }}
	\renewcommand{\algorithmicforall}{\textbf{foreach }}

\section{HostDBEvents}
\begin{lstlisting}[language=Python]
interface HostDBEvents {

	# Host entntry
	######################################################################
	# ------------------------------------------------------------------ #
	# | IPv6 address | MAC | DPID | Portnum | Timeout | last refreshed | #
	# ------------------------------------------------------------------ #
	######################################################################
	
	# "hostentry" is a list containing values specified as above

	void hostAdded(hostentry);
	
	void hostUpdated(updatedentry, oldentry)
	
	void hostremoved(hostentry)
	
}
\end{lstlisting}

\section{HostDB}
\begin{lstlisting}[language=Python]
class HostDB {

  # Host_entry 
  ######################################################################
  # ------------------------------------------------------------------ #
  # | IPv6 address | MAC | DPID | Portnum | Timeout | last refreshed | #
  # ------------------------------------------------------------------ #
  ######################################################################

	# "hostentry" means list containing values as specified above
	
	enum index {IPV6, MAC, DPID, PORTNUM, TIMEOUT, LASTREFRESHED}

	List hostentries = []	  # list of hostentry
	
	HostDBEvents eventhandler
	
	############################################################
	# it perform searching based on searchby (index in the ho- #
	#  st-entry) and return all entries matches with the value #
	############################################################
	def getHostEntries(enum index searchby, unsigned long value);
	
	##########################################################
	#  it perform searching based on searchby. When it encou #
	# ters first match it return the value based on returnby #
	##########################################################
	def getHostEntryValue(enum index searchby, 
			      unsigned long value, 
			      enum index returnby);
	
	
	def addHost(unsigned long ipv6, 
		    unsigned long mac, 
		    unsigned long dpid, 
		    unsigned long portnum, 
		    unsgined long timeout);
	
	
	def updateHost(enum index [] searchby, 
		       unsigned long [] values, 
		       unsigned long [] entryvalues);
	
	def removeHost(enum index searchby, unsigned long value);
	
	def refreshHostEntry(enum index searchby, unsigned long value);
	
}
\end{lstlisting}

\begin{algorithm}
	\caption{getHostEntries(enum index searchby, unsigned long value)}
	\begin{algorithmic}[1]
		\STATE $entries = []$
		\FORALL {$entry\ \textbf{in}\ hostentries$}
			\IF {$entry[searchby] == value$}
				\STATE $entries.append(entry)$
			\ENDIF
		\ENDFOR
	\end{algorithmic}
\end{algorithm}

\newpage

\begin{algorithm}
	\caption{getHostEntryValue(enum index searchby,} \\
	\qquad \qquad \qquad \qquad \qquad \ \ unsigned long value, \\
	\qquad \qquad \qquad \qquad \qquad \ \ enum index returnby)
	\begin{algorithmic}[1]
		\FORALL{$entry\ \textbf{in}\ hostentries$}
			\IF{$entry[searchby] == value$}
				\RETURN $entry[returnby]$
			\ENDIF
		\ENDFOR
		\RETURN $0$
	\end{algorithmic}
\end{algorithm}


\begin{algorithm}
	\caption{addHost(unsigned long ipv6,}
		unsigned\ long\ mac,\\
		\qquad \qquad \qquad unsigned\ long\ dpid,\\
		\qquad \qquad \qquad unsigned\ long\ portnum,\\
		\qquad \qquad \qquad unsgined\ long\ timeout);
	\begin{algorithmic}[1]
		\STATE $entry = [ipv6, mac, dpid, portnum, timeout, currenttime]$
		\STATE $hostentries.append(entry)$
		\STATE $eventhandler.hostAdded(entry)$
	\end{algorithmic}
\end{algorithm}

\begin{algorithm}
	\caption{removeHost(enum index searchby, unsigned long value)}
	\begin{algorithmic}[1]
		\STATE $entries = []$
		\FORALL {$entry\ \textbf{in}\ hostentries$}
			\IF {$entry[searchby] == value$}
				\STATE $eventhandler.hostRemoved(entry)$
			\ELSE
				\STATE $entries.append(entry)$
			\ENDIF
		\ENDFOR
		\STATE $hostentries = entries$
	\end{algorithmic}
\end{algorithm}

\newpage

\begin{algorithm}
	\caption{def updateHost(enum index [] searchby,}\\ 
	\qquad \qquad \qquad \qquad \ \ \ unsigned long [] values,\\
	\qquad \qquad \qquad \qquad \ \ \ unsigned long [] entryvalues)
	\begin{algorithmic}[1]
		\STATE $searchindex = 0$
		\STATE $valueindex = 0$
		\FORALL {$entry\ \textbf{in}\ hostentries$}
			\FORALL {$sby\ \textbf{in}\ searchby$}
				\STATE $index = 0$
				\IF{$entry[sby] \neq searchby[index]$}
					\STATE $\textbf{break}$
				\ENDIF
				\STATE $index\ +=\ 1$
			\ENDFOR
			\IF {$index == searchby.length$}
				\STATE $oldentry = entry.clone()$
				\FOR {$index = 0\ \TO\ 6$}
					\IF {$index == searchindex$}
						\STATE $searchindex\ +=\ 1$
					\ELSE
						\STATE $entry[index] = values[valueindex]$
						\STATE $valueindex\ +=\ 1$
					\ENDIF
				\ENDFOR
				\STATE $eventhandler.hostUpdated(oldentry, entry)$
			\ENDIF
		\ENDFOR
	\end{algorithmic}
\end{algorithm}

\lstdefinestyle{mystyle}{
	breakatwhitespace=false,
	keepspaces=true	
}
\lstset{style=mystyle}

\section{Address}
\begin{lstlisting}[language=Python]
class Address {
	
	List multicastaddresses; # list contains IPv6 multicast, 
				 # MAC addresses pair
	
	List addresses;		 # list contains IPv6, MAC adresses pairs
	
	List defaultIPs;
	
	unsigned long defaultGateway;
	
	def addNewAddress(unsigned long ipv6, unsigned long mac);
	
	def addNewMulticastAddress(unsigned long ipv6, unsigned long mac);
	
	def removeAddress(unsigned long ipv6);
	
	########################################################
	# It matches the given value against the addresses and #
	# multicastaddresses object based on value given in va-#
	# luetype. If it matched any address it return true    #
	# else return false                                    #
	########################################################
	def matches(unsigned long value, int valuetype);
	
	########################################################
	# It returns the mac address assosiated with the given #
	# ipv6 address if it found it returns mac address else #
	# it returns 0                                         #
	########################################################
	def getMAC(unsigned long ipv6);
	
	#########################################################
	# It returns the ipv6 address assosiated with the given #
	# mac address if it found it returns ipv6 address else  #
	# it returns 0                                          #
	#########################################################
	def getIP(unsigned long mac);
	
}
\end{lstlisting}

\begin{algorithm}
	\caption{matches(unsigned long value, int valuetype)}
	\begin{algorithmic}[1]
		\STATE $addrlist = None$
		\IF {$valuetype == 0\ \AND\ value == MULTICASTIPv6ADDRESS$}
			\STATE $addrlist = multicastaddresses$
		\ELSE
			\STATE $addrlist = addresses$
		\ENDIF
		\FOR {$addrpair\ \textbf{in}\ addrlist$}
			\IF{$addrpair[valuetype] == value$}
				\RETURN \TRUE
			\ENDIF
		\ENDFOR
		\RETURN \FALSE
	\end{algorithmic}
\end{algorithm}

\section{ICMPv6Handler}
\begin{lstlisting}[language=Python]
class ICMPv6Handler {

	Address myaddress;	# Addresses are represented by the object,
				# are considered as its own addresess
	HostDb hosts;
	
	LocalNetwork network;
	
	def process(PacketInEvent event);
	
	def _handle_ICMPv6_Echo_Request(Ethernet ethpkt, 
					unsigned long dpid, 
					unsigned short inport);
	
	def _handle_ICMPv6_Neighbour_Solicitation(Ethernet ethpkt, 
						  unsigned long dpid, 
						  unsigned short inport);
	
	def isHostRechable(unsigned long targetip, 
			   unsigned long dpid = None, 
			   int port = None);

}
\end{lstlisting}

\begin{algorithm}
	\caption{process(PacketInEvent event)}
	\begin{algorithmic}[1]
		\STATE $ethpkt = event.packet$
		\STATE $ipv6 = ethpkt.find("ipv6")$
		\STATE $icmp = ipv6.find("icmpv6")$
		\IF {($icmppkt.type == 128\ \OR\ icmppkt.type == 144$) \AND \\ $\ \ \ ipv6.destination\ \textbf{is}\ \textbf{not}\ IPv6\_MULTICAST\_ADDRESS$}
			\IF{$myaddress.matches(ipv6.destination)$}
				\STATE $event.reason = OFPP\_CONTROLLER$
				\STATE $\_handle\_ICMPv6\_Echo\_Request(ethpkt, event.dpid, event.inport)$
			\ENDIF
		\ELSIF{$icmppkt.type == 133$}
			\IF{$myaddress.matches(icmp.target)$}
				\STATE $event.reason = OFPP\_CONTROLLER$
				\STATE $\_handle\_ICMPv6\_Neighbour\_Solicitation(ethpkt, event.dpid, event.inport)$
			\ENDIF
		\ELSIF{$icmppkt.type == 144$ \\ $\ \ \ \ \ \ \ \ \ \ \ \ \ \ ipv6.destination\ \textbf{is}\ ALL\_HOME\_AGENTS\_MULTICAST\_ADDRESS$}
			\STATE $event.reason = OFPP\_CONTROLLER$
			\STATE $\_handle\_ICMPv6\_Echo\_Request(ethpkt, event.dpid, event.inport)$
		\ENDIF
	\end{algorithmic}
\end{algorithm}

\begin{algorithm}
	\caption{\_handle\_ICMPv6\_Echo\_Request(Ethernet ethpkt,}\\
	\qquad \qquad \qquad \qquad \qquad \qquad \qquad \qquad unsigned long dpid,\\
	\qquad \qquad \qquad \qquad \qquad \qquad \qquad \qquad unsigned short inport)\\
	\begin{algorithmic}
		\STATE $ipv6pkt = ethpkt.find("ipv6")$
		\IF{$ipv6.destination \textbf{is} MULTICAST\_IPv6\_ADDRESS$}
			\STATE $srcip = address.defaultPublicIP()$
			\STATE $srcmac = address.getMAC(dstip)$
		\ELSE
			\STATE $srcip = ipv6pkt.destination$
			\STATE $srcmac = ethpkt.destination$
		\ENDIF
		\STATE $pkt = Ethernet.createPacket(srcmac, ethpkt.source)$
		\STATE $ipv6 = IPv6.createPacket(srcip, ipv6.source)$
		\STATE $icmppkt = ipv6pkt.find("icmpv6")$
		\IF{$icmppkt.type == 128\ \AND\ icmppkt.code == 0$}
			\STATE $icmp = ICMPv6.createEchoReply(icmppkt.Identifier, icmppkt.SequenceNo)$
		\ELSIF {$icmppkt.type == 144\ \AND\ icmppkt.code == 0$}
			\STATE $defaultIPs = address.defaultIPs.clone().remove(srcip)$
			\STATE $icmp = ICMPv6.createHomeAgentDiscoverReply(icmppkt.Identifier, defaultIPs)$
		\ENDIF
		\STATE $ipv6.data = icmp$
		\STATE $pkt.data = ipv6$
		\STATE $network.sendPacket(pkt,dpid, inport)$
	\end{algorithmic}
\end{algorithm}

\begin{algorithm}
	\caption{\_handle\_ICMPv6\_Neighbour\_Solicitation(Ethernet ethpkt,}\\
	\qquad \qquad \qquad \qquad \qquad \qquad \qquad \qquad \qquad \qquad unsigned long dpid,\\
	\qquad \qquad \qquad \qquad \qquad \qquad \qquad \qquad \qquad \qquad unsigned short inport)\\
	\begin{algorithmic}[1]
		\STATE $ipv6pkt = ethpkt.find("ipv6")$
		\STATE $icmppkt = ipv6pkt.find("icmpv6")$
		\STATE $srcip = icmppkt.target()$
		\STATE $srcmac = myaddress.getMAC(srcip)$
		\IF {$ipv6pkt.source\ \textbf{is}\ \textbf{UNSPECIFIED}$}
			\STATE $dstip = ALL\_NODES\_MULTICAST\_ADDRESS$
			\STATE $dstmac = ALL\_NODES\_MULTICAST\_ETHERNET\_ADDRESS$
		\ELSE
			\STATE $dstip = ipv6pkt.source$
			\STATE $dstmac = icmppkt.sourceLinkLayerAddress$
		\ENDIF
		\STATE $pkt = Ethernet.createPacket(srcmac, dstmac)$
		\STATE $ipv6 = IPv6.createPacket(srcip, dstip)$
		\STATE $ipv6.data = ICMPv6.createNeighbourAdvertisement(srcip, srcmac, flag = 'R')$
		\STATE $pkt.data = ipv6$
		\STATE $network.sendPacket(pkt, dpid, inport)$
	\end{algorithmic}
\end{algorithm}

%\begin{algorithm}
%	\caption{\_handle\_ICMPv6\_Echo\_Request(Ethernet ethpkt)}
%	\begin{algorithmic}[1]
%		\STATE $ipv6pkt = ethpkt.find("ipv6")$
%		\IF{$myaddress.matches(ipv6.destination)$}
%			\IF{$ipv6.destination \textbf{is} MULTICAST\_IPv6\_ADDRESS$}
%				\STATE $srcip = address.defaultIP[0]$
%				\STATE $srcmac = address.getMAC(dstip)$
%			\ELSE 
%				\STATE $srcip = ipv6pkt.destination$
%				\STATE $srcmac = ethpkt.destination$
%			\ENDIF
%			\STATE $pkt = Ethernet.createPacket(srcmac, ethpkt.source)$
%			\STATE $ipv6 = IPv6.createPacket(srcip, ipv6.source)$
%			\STATE $icmppkt = ipv6pkt.find("icmpv6")$
%			\STATE $hostentries = hosts.getHostentries(IPV6, ipv6.destinationIP)$
%			\IF{$icmppkt.type == 128\ \AND\ icmppkt.code == 0$}
%				\STATE $icmp = ICMPv6.createEchoReply(icmppkt.Identifier, icmppkt.SequenceNo)$
%			\ELSIF {$icmppkt.type == 144\ \AND\ icmppkt.code == 0$}
%				\STATE $defaultIPs = address.defaultIPs.clone().remove(srcip)$
%				\STATE $icmp = ICMPv6.createHomeAgentDiscoverReply(icmppkt.Identifier, defaultIPs)$
%			\ENDIF
%			\STATE $ipv6.data = icmp$
%			\STATE $pkt.data = ipv6$
%			\STATE $network.sendPacket(pkt, hostentries[0][DPID], hostentries[0][PORTNUM])$
%		\ENDIF
%	\end{algorithmic}
%\end{algorithm}

\begin{algorithm}
	\caption{isHostRechable(unsigned long targetip,} \\
	\qquad \qquad \qquad \qquad \ \ \ unsigned long dpid = None, \\
	\qquad \qquad \qquad \qquad \ \ \ int port = None)
	\begin{algorithmic}[1]
		\STATE $srcmac = myaddress.getMAC(myaddress.defaultIP)$
		\STATE $hostentries = hosts.getHostEntries(MAC, srcmac)$
		\STATE $hostentry = \textbf{None}$
		\STATE $srcip = \textbf{None}$
		\FORALL {$entry\ \textbf{in}\ hostentries$}
			\IF {$entry[IP]\ \textbf{is}\ type(targetip)$}
				\STATE $hostentry = entry$
				\STATE $srcip = entry[IP]$
				\STATE $\textbf{break}$
			\ENDIF
		\ENDFOR
		\IF {$dpid\ \textbf{is}\ \textbf{None}$}
			\STATE $dpid = hostentry[DPID]$
			\STATE $port = hostentry[PORTNUM]$
		\ENDIF
		\IF {$srcip\ \textbf{is}\ \NOT\ \textbf{None}$}
			\STATE $waittime = 1$
			\STATE $ethpkt = Ethernet.createPacket(srcmac, ALL\_NODES\_MULTICAST\_ETHERNET\_ADDRESS)$
			\STATE $ipv6 = IPv6.createPacket(srcip, ALL\_NODES\_MULTICAST\_ADDRESS)$
			\FOR {$i = 1\ \TO\ 3$}
				\STATE $currenttime = Date.getInstance()$
				\STATE $hostentry[LASTREFRESHED] = currenttime$
				\STATE $icmp = ICMPv6.createNeighbourSolicitation(srcmac, targetip)$
				\STATE $ipv6.data = icmp$
				\STATE $ethpkt.data = ipv6$
				\STATE $network.sendPacket(ethpkt, dpid, port)$
				\STATE $wait(waittime)$
				\STATE $currenttime = Date.getInstance()$
				\IF {$abs(hostentry[LASTREFRESHED] - currenttime) \leq waittime$}
					\RETURN \TRUE
				\ELSE
					\STATE $waittime = waittime \times 2$
				\ENDIF
			\ENDFOR
			\RETURN \FALSE
		\ELSE
			\RETURN \FALSE
		\ENDIF
		
	\end{algorithmic}
\end{algorithm}

%\begin{algorithm}
%	\caption{\_handle\_ICMPv6\_Neighbour\_Solicitation(Ethernet ethpkt)}
%	\begin{algorithmic}[1]
%		\STATE $ipv6pkt = ethpkt.find("ipv6")$
%		\STATE $icmppkt = ipv6pkt.find("icmpv6")$
%		\IF {$myaddress.matches(icmppkt.target)$}
%			\STATE $srcip = icmppkt.target()$
%			\STATE $srcmac = myaddress.getMAC(srcip)$
%			\IF {$ipv6pkt.source\ \textbf{is}\ \textbf{UNSPECIFIED}$}
%				\STATE $dstip = ALL\_NODES\_MULTICAST\_ADDRESS$
%				\STATE $dstmac = ALL\_NODES\_MULTICAST\_ETHERNET\_ADDRESS$
%			\ELSE
%				\STATE $dstip = ipv6pkt.source$
%				\STATE $dstmac = icmppkt.sourceLinkLayerAddress$
%			\ENDIF
%			\STATE $hostentryies = hosts.getHostEntries(MAC, icmppkt.sourceLinkLayerAddress)$
%			\STATE $pkt = Ethernet.createPacket(srcmac, dstmac)$
%			\STATE $ipv6 = IPv6.createPacket(srcip, dstip)$
%			\STATE $ipv6.data = ICMPv6.createNeighbourAdvertisement(srcip, srcmac, flag = 'R')$
%			\STATE $pkt.data = ipv6$
%			\STATE $network.sendPacket(pkt, hostentries[0][DPID], hostentries[0][PORTNUM])$
%		\ENDIF
%	\end{algorithmic}
%\end{algorithm}

\section{HomeAgent}
\begin{lstlisting}[language=Python]
class HomeAgent {
	
	List mobileagents;	# list contains ip, mac 
				#addresses pair of mobile nodes
	
	ICMPv6Handler ihandler;
	
	def _handle_Binding_Updates(Ethernet ethpkt);
	
}
\end{lstlisting}

\begin{algorithm}
	\caption{\_handle\_Binding\_Updates(PacketInEvent event)}
	\begin{algorithmic}
		\IF {$alternateCareofAddrsExist(ippkt)$}
			\STATE $caddrs = extractAltCareofAddresses(ippkt)$
		\ELSE 
			\STATE $caddrs = new\ List(ippkt.source)$
		\ENDIF
		\STATE $homeaddr = getHomeaddrFromHomeaddrOption(ippkt)$
		\STATE $mnReachable = ihandler.isHostRechable(homeaddr, event.dpid, event.port)$
		\STATE $lifetime = extractLifeTime(ippkt)$
		%\IF {$lifetime > 0\ \AND\ \NOT\ mnReachable$}
		%\ELSE 
		%\ENDIF
	\end{algorithmic}
\end{algorithm}

%\begin{algorithm}
%	\caption{\_handle\_Binding\_Updates(IPv6 ippkt)}
%	\begin{algorithmic}[1]
%		\STATE $homeaddress = extractHomeAddress(ippkt)$
%		\STATE $careofaddress = extractAltCareofAddress(ippkt)$
%		\IF {$careofaddress\ \textbf{is}\ \textbf{None}$}
%			\STATE $careofaddress = ippkt.source$
%		\ENDIF
%		\STATE $lifetime = extractLifeTime(ippkt)$
%		\IF {$lifetime == 0$}
%			\STATE $rechable = icmphandler.isHostRechable()$
%		\ELSE 
%		\ENDIF
%	\end{algorithmic}
%\end{algorithm}

\section{LocalNetwork}	
\begin{lstlisting}[language=Python]
class LocalNetwork {

	Map<long, Switch> switchlist 	#Each object in the list represents 
					#one physical switch in the network
	Graph topology;			#Topology of the network
	Graph shortestpaths;		#All pairs shortest paths
	
	#############################################
	# The function handles ConnectionUp events. #
	# This event raised when the controller has #
	# detected a new switch in the network.     #
	#############################################
	def _handle_ConnectionUp(Event event);
	
	##################################################
	# The function handles ConnectionDown events.	 #
	# This event raised when the controller has fa-	 #
	# iled to detect existing switch in the network. #
	# It removes the corresponding switch object and #
	# edges connected to the object.                 #
	##################################################
	def _handle_ConnectionDown(Event event);
	
	##################################################
	# The fucntion handles link event which is rais- # 
	# ed on detection of link between two switches.  #
	# if link_added flag has set it adds bidirecti-  #
	# onal link between two switches in the graph.   #
	# else it remove bidirection link between two	 #
	# switches in graph.                             #
	##################################################
	def _handle_LinkEvent(LinkEvent event);

	#################################################
	#  The event raised when the controller has re- #
	# ceived the PortStatus message. The message    #
	# contains configuration of a specific port of  #
	# a switch. On receiving the message, it add    #
	# portnumber, mac address pair to corresponding #
	# switch object.                                #
	#################################################
	def _handle_PortStatus(PortStatusEvent event);
	
	################################################
	# The function returns the switch object       #
	################################################
	def getSwitch(String dpid);
	
	##################################################
	# The function returns the set of Switch objects #
	##################################################
	def getPath(long srcDPID, long dstDPID);
	
	#############################################
	# The function calculate all pairs shortest #
	# paths from the topology object.           #
	#############################################
	def floyd_warshall_algorithm();
	
}
	\end{lstlisting}
	
	
	\begin{algorithm}
		\caption{\_handle\_ConnectionUp(Event event)}
		\begin{algorithmic}[1]
			\STATE $switch \leftarrow new Switch(event.dpid, event.connection, "switch.conf")$
			\STATE $switchlist.append(switch)$
			\STATE $topology.addNode(event.dpid)$
		\end{algorithmic}
	\end{algorithm}
	
	\begin{algorithm}
		\caption{\_handle\_ConnectionDown(Event event)}
		\begin{algorithmic}[1]
			\STATE $adjlist \leftarrow topology.getAdjList(event.dpid)$
			\STATE $switch \leftarrow switchlist.get(event.dpid)$
			\STATE $switch.deleted \leftarrow \TRUE$
			\STATE $switchlist.remove(event.dpid)$
			\STATE $topology.removeNode(event.dpid)$
			\FORALL {dpid \textbf{in} adjlist}
				\STATE $topology.removeLink(dpid, event.dpid)$
			\ENDFOR
			\STATE $floyd\_warshall\_algorithm()$
		\end{algorithmic}
	\end{algorithm}
	
	\begin{algorithm}
		\caption{\_handle\_LinkEvent(LinkEvent event)}
		\begin{algorithmic}[1]
			\IF {event.linkAdded}
				\IF {\NOT topology.hasEdge(event.dpid, event.dpid2)}
					\STATE $topology.addLink(event.dpid1, event.dpid2, event.port1)$
					\STATE $topology.addLink(event.dpid2, event.dpid2, event.port2)$
					\STATE $floyd\_warshall\_algorithm()$
				\ENDIF
			\ELSE
				\STATE $topology.removeLink(event.dpid1, event.dpid2)$
				\STATE $topology.removeLink(evnet.dpid2, event.dpid1)$
				\STATE $floyd\_warshall\_algorithm()$
			\ENDIF
		\end{algorithmic}
	\end{algorithm}
	
	\begin{algorithm}
		\caption{\_handle\_PortStatus(PortStatusEvent event)}
		\begin{algorithmic}[1]
			\STATE $switch \leftarrow switchlist.get(event.dpid)$
			\STATE $switch.addPort(event.portnum, event.mac, event.porttype)$
		\end{algorithmic}
	\end{algorithm}
	
	
	\begin{algorithm}
		\caption{getPath(long srcDPID, long dstDPID)}
		\begin{algorithmic}[1]
			\STATE $path \leftarrow []$
			\REPEAT
				\STATE $path.append(switchlist.get(srcDPID))$
				\STATE $srcDPID \leftarrow shortestpaths.getNextNode(srcDPID, dstDPID)$
			\UNTIL{srcDPID = dstDPID}
			\RETURN $path$
		\end{algorithmic}
	\end{algorithm}
	
\section{ICMPv6Handler}
\begin{lstlisting}[language=Python]
class ICMPv6Handler {
	
	Address myaddress;	# addresses represented by this object 
				# is considered as its own addresses
	
	HostDb hostdb;		# The object contains list hosts and its 
				#positions in the network
	
	##############################################################
	# The definition uses neighbour solicitation message to refr-#
	# esh entries in the hostdb object as well as send neighbour #
	# advertisement for addresess represented by myaddress object#
	##############################################################
	def _handle_Neighbour_Solicitation(icmpv6msg);
	
	################################################
	# The definition uses neighbour advertisement  #
	# messages for refresh host entries hostdb.    #
	################################################
	def _handle_Neighbour_Advertisement(icmpv6msg);
	
	################################################
	# The definition uses router advertisement to  #
	# learn edge routers connected to the network. #
	################################################
	def _handle_Router_Advertisement(icmpv6msg);
	
	##############################################################
	# The definition sends neighbour solicitation message to the #
	# target ip with source IP is default ip in myaddress object #
	#  and its corresponding mac address is source mac. It sends #
	# message three times in exponential backoff time, if it re- #
	#  ceived reply at lease one time it returns true else false #
	##############################################################
	def isHostReachable(targetIP);
	
}
\end{lstlisting}

\begin{algorithm}
	\caption{isHostReachable(targetIP)}
	\begin{algorithmic}[1]
		\STATE $srcip \leftarrow myaddress.defaultIP$
		\STATE $srcmac \leftarrow myaddress.getMac(srcip)$
		\STATE $waittime \leftarrow 1$
		\STATE $hostentry \leftarrow hostdb.getEntrybyIP(targetIP)$
		\FOR {$i = 0\ \TO\ 3$}
			\STATE $time \leftarrow new\ Date()$
			\STATE $hostentry[lastrefreshed] \leftarrow time$
			\STATE $sendNeighbourSolicitation(srcip, srcmac, targetIP)$
			\STATE $wait(waittime)$
			\IF {$abs(hostentry[lastrefreshed] - time) \leq waittime$}
				\RETURN \TRUE
			\ELSE
				\STATE $waittime \leftarrow waittime \times 2$
			\ENDIF
		\ENDFOR
		\RETURN \FALSE
	\end{algorithmic}
\end{algorithm}
	
\end{document}